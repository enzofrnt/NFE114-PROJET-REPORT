\section{UML — Diagramme de classes}

Diagramme des classes PHP du modèle (attributs, méthodes et relations).

\begin{figure}[H]
\centering
\begin{tikzpicture}[
  font=\small,
  node distance=2.2cm and 3.2cm,
  umlclass/.style={
    draw,
    rectangle split,
    rectangle split parts=3,
    align=left,
    text width=4.4cm
  },
  every path/.style={-{Stealth[length=2.5pt]}}
]

% ======================
% Database (haut)
% ======================
\node[umlclass, xshift=-1.5cm] (db) {
  \textbf{Database}
  \nodepart{second}
  \textit{instance : self}\\
  \textit{pdo : PDO}
  \nodepart{third}
  + getInstance() : self\\
  + pdo() : PDO
};

% ======================
% Cours (centre)
% ======================
\node[umlclass, below=of db, xshift=-2.5cm] (cours) {
  \textbf{Cours}
  \nodepart{second}
  \textit{id, coach\_id, salle\_id : int}\\
  \textit{nom, description, type\_id : string}\\
  \textit{date\_heure\_debut : datetime}\\
  \textit{duree\_minutes, capacite\_max : int}
  \nodepart{third}
  + getAll(typeId?: string) : array\\
  + getById(id: int) : ?array\\
  + getByCoach(coachId: int) : array\\
  + getDisponibilite(coursId: int) : array\\
  + create(data: array) : int\\
  + update(id: int, data: array) : bool\\
  + delete(id: int) : bool\\
  + getStats() : array
};

% ======================
% Métier – gauche
% ======================
\node[umlclass, left=of cours] (adh) {
  \textbf{Adherent}
  \nodepart{second}
  \textit{id : int}\\
  \textit{nom, prenom, email : string}
  \nodepart{third}
  + getAll() : array
};

% ======================
% Métier – droite
% ======================
\node[umlclass, right=of cours, xshift=-1.5cm] (coach) {
  \textbf{Coach}
  \nodepart{second}
  \textit{id : int}\\
  \textit{nom, prenom, email : string}
  \nodepart{third}
  + getAll() : array
};

% ======================
% Bas – associations
% ======================
\node[umlclass, below=of adh] (res) {
  \textbf{Reservation}
  \nodepart{second}
  \textit{id, cours\_id, adherent\_id : int}\\
  \textit{created\_at : datetime}
  \nodepart{third}
  + create(adherentId: int, coursId: int) : array\\
  + cancel(reservationId: int) : bool\\
  + getByAdherent(adherentId: int) : array\\
  + getParticipants(coursId: int) : array
};

\node[umlclass, below=of coach] (salle) {
  \textbf{Salle}
  \nodepart{second}
  \textit{id : int}\\
  \textit{nom : string}\\
  \textit{capacite\_max : int}
  \nodepart{third}
  + getAll() : array
};

% ======================
% Relations
% ======================

% Database
\draw[dashed] (db) -- node[right, font=\scriptsize] {utilise PDO} (cours);

% Cours
\draw (coach) -- node[above, font=\scriptsize] {1 anime *} (cours);
\draw (salle) -- node[right, font=\scriptsize] {1 accueille *} (cours);

% Réservations
\draw (adh) -- node[left, font=\scriptsize] {* effectue} (res);
\draw (res) -- node[below, font=\scriptsize] {* concerne} (cours);

\end{tikzpicture}
\caption{Diagramme de classes UML — Modèle PHP (FitBooking)}
\end{figure}

\subsection{Légende et remarques}
\begin{itemize}[leftmargin=*]
  \item \textbf{Database} : singleton (connexion PDO partagée), utilisé par tous les modèles.
  \item Attributs : cohérents avec le MLD (tables ADHERENT, COACH, SALLE, COURS, RESERVATION) ; en PHP les modèles accèdent aux données via requêtes SQL.
  \item \textbf{Cours} : gestion du planning, filtrage par type, CRUD admin, statistiques (taux de remplissage).
  \item \textbf{Reservation} : création avec transaction et verrou (\texttt{FOR UPDATE}), contrôle capacité et unicité (\texttt{cours\_id}, \texttt{adherent\_id}).
  \item Relations : un \textbf{Cours} est animé par un \textbf{Coach} (1--*) et a lieu dans une \textbf{Salle} (1--*) ; une \textbf{Reservation} lie un \textbf{Adherent} (*) à un \textbf{Cours} (*).
\end{itemize}
